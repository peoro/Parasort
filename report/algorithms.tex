\section{Objectives, assumptions and algorithms}
\label{assumptions}
In the first part of our work we put a lot of efforts in deeply understanding the state of the art of parallel sorting algorithms. The first thing we noticed is the lack of an unified theory that asserts whether an algorithm is better than another; this is an obvious consequence of the absence of realistic cost models for parallel algorithms and architectures (e.g. both $PRAM$ and the more refined $LogP$ are too superficials to be considered significant. Second thing, the literature is rich of parallel algorithms conceived and engineered for \textit{specific machines}, so to be able to exploit peculiar features of the machine itself, e.g. the structure of the interconnection network. Just as examples we can cite~\cite{CSPA, CSPA2}, but with some googling we can really find a lot of papers or report on these themes. This approach is not obviously good because of at least two facts: the high designing complexity and the non-portability. Hence, in our project, even if the practical performance analysis will be done on a specific architecture, the parallel algorithms will \textit{not} be designed for that specific machine. In some sense, our approach is more close to~\cite{NPSA}.

We list the sorting algorithms that we are going to parallelize, all of them based on the Divide$\&$Conquer paradigm. Notice that, in general, the parallelization consist of ''exporting'' either the divide or the conquer phase to the ''process level''. Just as an example, in parallel Mergesort the distribution phase will be made between the processes of the parallel algorithm. 
\begin{enumerate}
\item Mergesort
\item K-Way Mergesort
\item Load-Balance Mergesort (and some variants) 
\item Quicksort
\item Bitonicsort
\item Bucketsort
\item Samplesort
\end{enumerate}
Examples of parallelization of $some$ of these algorithms can be find in literature. For instance, it is easy to find parallel versions of Mergesort, Quicksort, Samplesort, Bitonicsort. On the other hand, we notice the lack (or at least the poor availability) of informations regarding K-Way Mergesort and Load-Balance Mergesort. In any case, as we will better explain later, we re-implemented all the algorithms. Indeed, all the MPI implementations we found were really poor in terms of quality of code and do not take into account the possibility of having to sort large data sets, a problem which requires a specific analysis. Once implemented, we will compare the performance of these algorithms in terms of their efficiency, scalability and completion time, by playing on some degrees of freedom like the parallelism degree and the size of the data sets.  
 
In this section we will summarize the assumptions for this project, emerged during the several meetings.
\begin{itemize}
	\item{The work we need to do is to compare parallel sorting algorithms and to study how they behave. We early understood that to achieve such goal we were forced to collaborate in some way. We had to define a common framework in order to have comparable results. If any of us computed times in different ways, these wouldn't have been comparable. If any of us used a different strategy to sort data sequentially, or different optimization to sort data in principal memory or in disk, results would have been worthless, since the comparison would have mixed the performance of a parallel sorting algorithm with the performance of something else that regards sequential computation. This, besides, allowed us to do a better work, having a well structured framework which can be extended to many other algorithms with little work.}
	\item{We are not interested to study how sequential algorithms behave. We care only about the parallel part of our algorithms. The sequential part is always implemented using the \textit{qsort} POSIX standard function, and it does not matter if it works on primary memory only, rather than on secondary one. Sequential and parallel sorting are completely disjoint operations, which can be studied separately.}
	\item{Some algorithms natively begin, or end, with data centralized in a single process, while others begin or end with data distributed among all processes. For example mergesort begins with data distributed: any process sorts its piece fo data, and send it to another node that takes care of merging it, in order to end with data centralized on a single process. Quicksort, at the opposite, start with centralized data; the first process splits data and sends it to the other, in order to end up with distributed sorted data. For this reason and to mantain some homogenity among the algorithms we chose to always begin and end with centralized data. The algorithms that begin or end with distributed data will take care of scattering or gathering data, studying these extra (ie: unneeded by the algorithm, but needed by our common algorithm-testbed) operations separately from the rest of the computation.}
	\item{Since our platform is a clusted made of multicore nodes, we know that communication patterns between our several processes may weight a lot on the final performances. We will be looking for heuristical approaches to map processes in order to study what scenario will result in best performances due to communications between cores on different nodes, or on different parts of the single node.}
\end{itemize}
