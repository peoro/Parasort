\subsection*{Terminology}
We briefly describe the terminology that we are going to use in this chapter. Our parallel algorithms have been implemented on top of MPI, thus they are composed by a set of $N$ processes, each of them having its unique identifier. According to the MPI terminology, we refer to a specific process by calling it $rank$ $x$, where $x \in N$ represents its identifier. The input sequence $S$ that has to be ordered is of size $n$. 

In order to define the \textit{efficiency} $\varphi$ of an algorithm, we have to introduce some parameters:
\begin{itemize}
\item $P$, the number of real processors of our machine; 
\item $m$, the number of steps of the algorithm;
\item $T_i$, the duration of step $i$ (in seconds);
\item $X_i$, a random variable which counts the number of real processors that contribute to the calculation during the step $i$.
\end{itemize}
Therefore, we can define the efficiency as:
\begin{center}
$\varphi = \frac{\sum_{i=0}^m \frac{E[X_i]}{P}}{m} = \frac{\sum_{i=0}^m E[X_i]}{P \times m} $
\end{center}
The efficiency is a formal tool to establish how much an algorithm exploits the parallelism degree of a machine. $\varphi \rightarrow 1$ means that the algorithm, in each step, tends to use all the available processor of the machine. Obviously, $\varphi \rightarrow 1$ does not means that the algorithm is ''good''; indeed, for such claim, the value of both $m$ and $T_i$ must be considered. For instance, we might have $\varphi \rightarrow 1$ and $T_i \rightarrow\infty$, meaning that each step requires a lot of time; or even worse, $m \rightarrow\infty$, meaning that the algorithm requires a lot of steps in which almost all the processors are involved.

As we said, a parallel sorting algorithm is composed of some steps of computation. During each step, a process can do basically three things, obviously not mutual exclusive each other:
\begin{itemize}
\item it can send data to other processes (the process is a \textit{sender} of that step)
\item it can receive data from other processes (the process is a \textit{receiver} of that step)
\item it can perform computation.
\end{itemize}
