\subsubsection{PCM}
$PCM$ is a cluster of shared memory machines, thus all considerations reguarding mapping of processes to cores made in~\ref{fram-intr} become matter of study. Even the official guide of \textit{mvapich2}, the version of MPI that we have used to exploit Infiniband, emphasizes the importance of process-to-core mapping: as shown in~\cite{MVAPICH2-MAPPING}, different allocations of processes to cores can have a significant impact on the cost of communications. In principle, an entire study could be dedicated to this topic, leading to a lot of possible mappings, each one based on its reasonable heuristic. Obviously, we had to limit our analysis to a subset of the most simple mappings. In particular, we have studied the following configurations:   
\begin{itemize}
\item \textbf{Sequential mapping}. Adjacent ranks mapped on adjacent \textit{cores}. E.g., given two CPUs each one with 4 cores and 8 MPI processes, rank 0 goes on the first core of the \textit{first} CPU, rank 1 goes on the second core of the \textit{first} CPU, ..., rank 5 goes on the first core of the \textit{second} CPU and so on.
\item \textbf{Interleaved mapping}. Adjacent ranks mapped on adjacent \textit{CPUs}. E.g., given two CPUs each one with 4 cores and 8 MPI processes, rank 0 goes on the first core of the \textit{first} CPU, rank 1 goes on the first core of the \textit{second} CPU, rank 2 goes on the second core of the \textit{first} CPU and so on.
\item \textbf{Algorithm-specific mapping}. We have also studied a few mapping based on the nature of a specific Sorting Algorithm, like \textit{Mergesort} and \textit{Quicksort}. For instance, a mapping could be designed to let the most critical part of a stencil to take place in shared memory rather than inter-node.  
\end{itemize}
In the following, we will consider only the \textit{sequential mapping} because we have practically experienced that collective communications (in particular, the \textit{gather}) are more cost-effective than for other mappings. 


\paragraph{Scalability of Sorting Algorithms}

\paragraph{Comparison between Sorting Algorithms}

\clearpage