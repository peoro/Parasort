\subsubsection{Quicksort}
Quicksort, like Mergesort, is a very popular sorting algorithm, the most widely used on sequential architectures. The algorithm has been implemented in a clean way, separating three logical phases:
\begin{itemize}
	\item{\textit{scattering} phase: processes that have some data partition it (according to a randomly-chosen pivot), and send it to waiting processes.}
	\item{sequential sorting phase: any process sorts its own partition with a sequential partition.}
	\item{\textit{gathering} phase: processes send sorted data back, to rebuild the whole sorted array.}
\end{itemize}
We will not spend any time talking about the sequential sorting phase, since it is in common with all the other algorithms.
The \textit{scattering} and the \textit{gathering} phases, instead, are a bit tricky; they are not the simple, \textit{primitive} scatter and gather offered by MPI, since they need to do some more computation: our custom scattering phase needs to choose a pivot, partition the array according to the pivot, and send a whole partition away, while the gather knows exactly where to place the data it is receiving, since such data is one of the partitions, which has been sorted.
We chose to implement our scattering phase as a binary three: rank 0 will partition its data and send the second partition to a second process, and this will go on recursively until all of our processes get their data. The idea of implementing a linear scattering has been discarded because it would not have allowed a parallelization of the partitioning, which is the most expensive operation of the scattering phase.
The gathering phase has instead been implemented in a linear way: each process sends its sorted data to rank 0 which will read the data directly in the final array, without needing a further merging operation.

\subsubsection*{Mapping virtual processors onto different cores}  